\section{Introduction}
\label{sec:introduction}
Each day people share personal information with their web
browser. Browsers are partially retaining this information in order to
become faster and more responsive. This data must, at some point, reside
in memory before being discarded on program exit. Therefore, we
believe that there are forensically relevant artifacts that can be
extracted from a memory dump of such a browser process.

Memory forensics is a heavily researched topic. Tools such as
Volatility\footnote{\url{https://www.volatilityfoundation.org/}} have
been developed that help with scanning and extraction of processes,
files, registries, dynamic libraries, etc. However, less work has been
done for single process dump analysis. Thus, there is a need for
research to show that there is valuable information to be found in
browser memory dumps and tooling that facilitates easy extraction of
said artifacts. Optimistically, even whole DOM trees can be
reconstructed for each open tab at the moment the memory image was
done.

Some people are becoming more privacy aware and try to hide their
online activities. However, there are also people using these same
anonymization techniques in order to hide illegal activity. The Tor
browser\footnote{\url{https://www.torproject.org/}} is a free software
browser for enabling anonymous communication. Its threat model
protects against adversaries on the network, hence making remote
analysis much harder or even impossible. For that reason, we believe
that a memory analysis approach is an interesting angle for the
forensic analysis of the Tor browser.  \\

For this purpose, we define the following research question and
sub-questions that arise from it:

\textit{Can we extract and perhaps reconstruct relevant Tor browser
  information from memory dumps?}  \\

Which leads to asking following sub-questions:
\begin{enumerate}
\item If the extraction of such data is possible, it is possible to
  facilitate or even automate this process?
\item Can we build a tool that makes extraction faster and easier?
\end{enumerate}

Since we are interested in the forensic significance of our research
we have to consider the quality of the extracted data. This leads to
another sub-question:
\begin{enumerate}[resume]
\item How sound is the data recovered given the context of the
  investigation?
\end{enumerate}
